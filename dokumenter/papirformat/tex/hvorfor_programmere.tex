\documentclass[11pt]{article}

    \usepackage[breakable]{tcolorbox}
    \usepackage{parskip} % Stop auto-indenting (to mimic markdown behaviour)
    
    \usepackage{iftex}
    \ifPDFTeX
    	\usepackage[T1]{fontenc}
    	\usepackage{mathpazo}
    \else
    	\usepackage{fontspec}
    \fi

    % Basic figure setup, for now with no caption control since it's done
    % automatically by Pandoc (which extracts ![](path) syntax from Markdown).
    \usepackage{graphicx}
    % Maintain compatibility with old templates. Remove in nbconvert 6.0
    \let\Oldincludegraphics\includegraphics
    % Ensure that by default, figures have no caption (until we provide a
    % proper Figure object with a Caption API and a way to capture that
    % in the conversion process - todo).
    \usepackage{caption}
    \DeclareCaptionFormat{nocaption}{}
    \captionsetup{format=nocaption,aboveskip=0pt,belowskip=0pt}

    \usepackage[Export]{adjustbox} % Used to constrain images to a maximum size
    \adjustboxset{max size={0.9\linewidth}{0.9\paperheight}}
    \usepackage{float}
    \floatplacement{figure}{H} % forces figures to be placed at the correct location
    \usepackage{xcolor} % Allow colors to be defined
    \usepackage{enumerate} % Needed for markdown enumerations to work
    \usepackage{geometry} % Used to adjust the document margins
    \usepackage{amsmath} % Equations
    \usepackage{amssymb} % Equations
    \usepackage{textcomp} % defines textquotesingle
    % Hack from http://tex.stackexchange.com/a/47451/13684:
    \AtBeginDocument{%
        \def\PYZsq{\textquotesingle}% Upright quotes in Pygmentized code
    }
    \usepackage{upquote} % Upright quotes for verbatim code
    \usepackage{eurosym} % defines \euro
    \usepackage[mathletters]{ucs} % Extended unicode (utf-8) support
    \usepackage{fancyvrb} % verbatim replacement that allows latex
    \usepackage{grffile} % extends the file name processing of package graphics 
                         % to support a larger range
    \makeatletter % fix for grffile with XeLaTeX
    \def\Gread@@xetex#1{%
      \IfFileExists{"\Gin@base".bb}%
      {\Gread@eps{\Gin@base.bb}}%
      {\Gread@@xetex@aux#1}%
    }
    \makeatother

    % The hyperref package gives us a pdf with properly built
    % internal navigation ('pdf bookmarks' for the table of contents,
    % internal cross-reference links, web links for URLs, etc.)
    \usepackage{hyperref}
    % The default LaTeX title has an obnoxious amount of whitespace. By default,
    % titling removes some of it. It also provides customization options.
    \usepackage{titling}
    \usepackage{longtable} % longtable support required by pandoc >1.10
    \usepackage{booktabs}  % table support for pandoc > 1.12.2
    \usepackage[inline]{enumitem} % IRkernel/repr support (it uses the enumerate* environment)
    \usepackage[normalem]{ulem} % ulem is needed to support strikethroughs (\sout)
                                % normalem makes italics be italics, not underlines
    \usepackage{mathrsfs}
    

    
    % Colors for the hyperref package
    \definecolor{urlcolor}{rgb}{0,.145,.698}
    \definecolor{linkcolor}{rgb}{.71,0.21,0.01}
    \definecolor{citecolor}{rgb}{.12,.54,.11}

    % ANSI colors
    \definecolor{ansi-black}{HTML}{3E424D}
    \definecolor{ansi-black-intense}{HTML}{282C36}
    \definecolor{ansi-red}{HTML}{E75C58}
    \definecolor{ansi-red-intense}{HTML}{B22B31}
    \definecolor{ansi-green}{HTML}{00A250}
    \definecolor{ansi-green-intense}{HTML}{007427}
    \definecolor{ansi-yellow}{HTML}{DDB62B}
    \definecolor{ansi-yellow-intense}{HTML}{B27D12}
    \definecolor{ansi-blue}{HTML}{208FFB}
    \definecolor{ansi-blue-intense}{HTML}{0065CA}
    \definecolor{ansi-magenta}{HTML}{D160C4}
    \definecolor{ansi-magenta-intense}{HTML}{A03196}
    \definecolor{ansi-cyan}{HTML}{60C6C8}
    \definecolor{ansi-cyan-intense}{HTML}{258F8F}
    \definecolor{ansi-white}{HTML}{C5C1B4}
    \definecolor{ansi-white-intense}{HTML}{A1A6B2}
    \definecolor{ansi-default-inverse-fg}{HTML}{FFFFFF}
    \definecolor{ansi-default-inverse-bg}{HTML}{000000}

    % commands and environments needed by pandoc snippets
    % extracted from the output of `pandoc -s`
    \providecommand{\tightlist}{%
      \setlength{\itemsep}{0pt}\setlength{\parskip}{0pt}}
    \DefineVerbatimEnvironment{Highlighting}{Verbatim}{commandchars=\\\{\}}
    % Add ',fontsize=\small' for more characters per line
    \newenvironment{Shaded}{}{}
    \newcommand{\KeywordTok}[1]{\textcolor[rgb]{0.00,0.44,0.13}{\textbf{{#1}}}}
    \newcommand{\DataTypeTok}[1]{\textcolor[rgb]{0.56,0.13,0.00}{{#1}}}
    \newcommand{\DecValTok}[1]{\textcolor[rgb]{0.25,0.63,0.44}{{#1}}}
    \newcommand{\BaseNTok}[1]{\textcolor[rgb]{0.25,0.63,0.44}{{#1}}}
    \newcommand{\FloatTok}[1]{\textcolor[rgb]{0.25,0.63,0.44}{{#1}}}
    \newcommand{\CharTok}[1]{\textcolor[rgb]{0.25,0.44,0.63}{{#1}}}
    \newcommand{\StringTok}[1]{\textcolor[rgb]{0.25,0.44,0.63}{{#1}}}
    \newcommand{\CommentTok}[1]{\textcolor[rgb]{0.38,0.63,0.69}{\textit{{#1}}}}
    \newcommand{\OtherTok}[1]{\textcolor[rgb]{0.00,0.44,0.13}{{#1}}}
    \newcommand{\AlertTok}[1]{\textcolor[rgb]{1.00,0.00,0.00}{\textbf{{#1}}}}
    \newcommand{\FunctionTok}[1]{\textcolor[rgb]{0.02,0.16,0.49}{{#1}}}
    \newcommand{\RegionMarkerTok}[1]{{#1}}
    \newcommand{\ErrorTok}[1]{\textcolor[rgb]{1.00,0.00,0.00}{\textbf{{#1}}}}
    \newcommand{\NormalTok}[1]{{#1}}
    
    % Additional commands for more recent versions of Pandoc
    \newcommand{\ConstantTok}[1]{\textcolor[rgb]{0.53,0.00,0.00}{{#1}}}
    \newcommand{\SpecialCharTok}[1]{\textcolor[rgb]{0.25,0.44,0.63}{{#1}}}
    \newcommand{\VerbatimStringTok}[1]{\textcolor[rgb]{0.25,0.44,0.63}{{#1}}}
    \newcommand{\SpecialStringTok}[1]{\textcolor[rgb]{0.73,0.40,0.53}{{#1}}}
    \newcommand{\ImportTok}[1]{{#1}}
    \newcommand{\DocumentationTok}[1]{\textcolor[rgb]{0.73,0.13,0.13}{\textit{{#1}}}}
    \newcommand{\AnnotationTok}[1]{\textcolor[rgb]{0.38,0.63,0.69}{\textbf{\textit{{#1}}}}}
    \newcommand{\CommentVarTok}[1]{\textcolor[rgb]{0.38,0.63,0.69}{\textbf{\textit{{#1}}}}}
    \newcommand{\VariableTok}[1]{\textcolor[rgb]{0.10,0.09,0.49}{{#1}}}
    \newcommand{\ControlFlowTok}[1]{\textcolor[rgb]{0.00,0.44,0.13}{\textbf{{#1}}}}
    \newcommand{\OperatorTok}[1]{\textcolor[rgb]{0.40,0.40,0.40}{{#1}}}
    \newcommand{\BuiltInTok}[1]{{#1}}
    \newcommand{\ExtensionTok}[1]{{#1}}
    \newcommand{\PreprocessorTok}[1]{\textcolor[rgb]{0.74,0.48,0.00}{{#1}}}
    \newcommand{\AttributeTok}[1]{\textcolor[rgb]{0.49,0.56,0.16}{{#1}}}
    \newcommand{\InformationTok}[1]{\textcolor[rgb]{0.38,0.63,0.69}{\textbf{\textit{{#1}}}}}
    \newcommand{\WarningTok}[1]{\textcolor[rgb]{0.38,0.63,0.69}{\textbf{\textit{{#1}}}}}
    
    
    % Define a nice break command that doesn't care if a line doesn't already
    % exist.
    \def\br{\hspace*{\fill} \\* }
    % Math Jax compatibility definitions
    \def\gt{>}
    \def\lt{<}
    \let\Oldtex\TeX
    \let\Oldlatex\LaTeX
    \renewcommand{\TeX}{\textrm{\Oldtex}}
    \renewcommand{\LaTeX}{\textrm{\Oldlatex}}
    % Document parameters
    % Document title
    \title{hvorfor\_programmere}
    
    
    
    
    
% Pygments definitions
\makeatletter
\def\PY@reset{\let\PY@it=\relax \let\PY@bf=\relax%
    \let\PY@ul=\relax \let\PY@tc=\relax%
    \let\PY@bc=\relax \let\PY@ff=\relax}
\def\PY@tok#1{\csname PY@tok@#1\endcsname}
\def\PY@toks#1+{\ifx\relax#1\empty\else%
    \PY@tok{#1}\expandafter\PY@toks\fi}
\def\PY@do#1{\PY@bc{\PY@tc{\PY@ul{%
    \PY@it{\PY@bf{\PY@ff{#1}}}}}}}
\def\PY#1#2{\PY@reset\PY@toks#1+\relax+\PY@do{#2}}

\expandafter\def\csname PY@tok@w\endcsname{\def\PY@tc##1{\textcolor[rgb]{0.73,0.73,0.73}{##1}}}
\expandafter\def\csname PY@tok@c\endcsname{\let\PY@it=\textit\def\PY@tc##1{\textcolor[rgb]{0.25,0.50,0.50}{##1}}}
\expandafter\def\csname PY@tok@cp\endcsname{\def\PY@tc##1{\textcolor[rgb]{0.74,0.48,0.00}{##1}}}
\expandafter\def\csname PY@tok@k\endcsname{\let\PY@bf=\textbf\def\PY@tc##1{\textcolor[rgb]{0.00,0.50,0.00}{##1}}}
\expandafter\def\csname PY@tok@kp\endcsname{\def\PY@tc##1{\textcolor[rgb]{0.00,0.50,0.00}{##1}}}
\expandafter\def\csname PY@tok@kt\endcsname{\def\PY@tc##1{\textcolor[rgb]{0.69,0.00,0.25}{##1}}}
\expandafter\def\csname PY@tok@o\endcsname{\def\PY@tc##1{\textcolor[rgb]{0.40,0.40,0.40}{##1}}}
\expandafter\def\csname PY@tok@ow\endcsname{\let\PY@bf=\textbf\def\PY@tc##1{\textcolor[rgb]{0.67,0.13,1.00}{##1}}}
\expandafter\def\csname PY@tok@nb\endcsname{\def\PY@tc##1{\textcolor[rgb]{0.00,0.50,0.00}{##1}}}
\expandafter\def\csname PY@tok@nf\endcsname{\def\PY@tc##1{\textcolor[rgb]{0.00,0.00,1.00}{##1}}}
\expandafter\def\csname PY@tok@nc\endcsname{\let\PY@bf=\textbf\def\PY@tc##1{\textcolor[rgb]{0.00,0.00,1.00}{##1}}}
\expandafter\def\csname PY@tok@nn\endcsname{\let\PY@bf=\textbf\def\PY@tc##1{\textcolor[rgb]{0.00,0.00,1.00}{##1}}}
\expandafter\def\csname PY@tok@ne\endcsname{\let\PY@bf=\textbf\def\PY@tc##1{\textcolor[rgb]{0.82,0.25,0.23}{##1}}}
\expandafter\def\csname PY@tok@nv\endcsname{\def\PY@tc##1{\textcolor[rgb]{0.10,0.09,0.49}{##1}}}
\expandafter\def\csname PY@tok@no\endcsname{\def\PY@tc##1{\textcolor[rgb]{0.53,0.00,0.00}{##1}}}
\expandafter\def\csname PY@tok@nl\endcsname{\def\PY@tc##1{\textcolor[rgb]{0.63,0.63,0.00}{##1}}}
\expandafter\def\csname PY@tok@ni\endcsname{\let\PY@bf=\textbf\def\PY@tc##1{\textcolor[rgb]{0.60,0.60,0.60}{##1}}}
\expandafter\def\csname PY@tok@na\endcsname{\def\PY@tc##1{\textcolor[rgb]{0.49,0.56,0.16}{##1}}}
\expandafter\def\csname PY@tok@nt\endcsname{\let\PY@bf=\textbf\def\PY@tc##1{\textcolor[rgb]{0.00,0.50,0.00}{##1}}}
\expandafter\def\csname PY@tok@nd\endcsname{\def\PY@tc##1{\textcolor[rgb]{0.67,0.13,1.00}{##1}}}
\expandafter\def\csname PY@tok@s\endcsname{\def\PY@tc##1{\textcolor[rgb]{0.73,0.13,0.13}{##1}}}
\expandafter\def\csname PY@tok@sd\endcsname{\let\PY@it=\textit\def\PY@tc##1{\textcolor[rgb]{0.73,0.13,0.13}{##1}}}
\expandafter\def\csname PY@tok@si\endcsname{\let\PY@bf=\textbf\def\PY@tc##1{\textcolor[rgb]{0.73,0.40,0.53}{##1}}}
\expandafter\def\csname PY@tok@se\endcsname{\let\PY@bf=\textbf\def\PY@tc##1{\textcolor[rgb]{0.73,0.40,0.13}{##1}}}
\expandafter\def\csname PY@tok@sr\endcsname{\def\PY@tc##1{\textcolor[rgb]{0.73,0.40,0.53}{##1}}}
\expandafter\def\csname PY@tok@ss\endcsname{\def\PY@tc##1{\textcolor[rgb]{0.10,0.09,0.49}{##1}}}
\expandafter\def\csname PY@tok@sx\endcsname{\def\PY@tc##1{\textcolor[rgb]{0.00,0.50,0.00}{##1}}}
\expandafter\def\csname PY@tok@m\endcsname{\def\PY@tc##1{\textcolor[rgb]{0.40,0.40,0.40}{##1}}}
\expandafter\def\csname PY@tok@gh\endcsname{\let\PY@bf=\textbf\def\PY@tc##1{\textcolor[rgb]{0.00,0.00,0.50}{##1}}}
\expandafter\def\csname PY@tok@gu\endcsname{\let\PY@bf=\textbf\def\PY@tc##1{\textcolor[rgb]{0.50,0.00,0.50}{##1}}}
\expandafter\def\csname PY@tok@gd\endcsname{\def\PY@tc##1{\textcolor[rgb]{0.63,0.00,0.00}{##1}}}
\expandafter\def\csname PY@tok@gi\endcsname{\def\PY@tc##1{\textcolor[rgb]{0.00,0.63,0.00}{##1}}}
\expandafter\def\csname PY@tok@gr\endcsname{\def\PY@tc##1{\textcolor[rgb]{1.00,0.00,0.00}{##1}}}
\expandafter\def\csname PY@tok@ge\endcsname{\let\PY@it=\textit}
\expandafter\def\csname PY@tok@gs\endcsname{\let\PY@bf=\textbf}
\expandafter\def\csname PY@tok@gp\endcsname{\let\PY@bf=\textbf\def\PY@tc##1{\textcolor[rgb]{0.00,0.00,0.50}{##1}}}
\expandafter\def\csname PY@tok@go\endcsname{\def\PY@tc##1{\textcolor[rgb]{0.53,0.53,0.53}{##1}}}
\expandafter\def\csname PY@tok@gt\endcsname{\def\PY@tc##1{\textcolor[rgb]{0.00,0.27,0.87}{##1}}}
\expandafter\def\csname PY@tok@err\endcsname{\def\PY@bc##1{\setlength{\fboxsep}{0pt}\fcolorbox[rgb]{1.00,0.00,0.00}{1,1,1}{\strut ##1}}}
\expandafter\def\csname PY@tok@kc\endcsname{\let\PY@bf=\textbf\def\PY@tc##1{\textcolor[rgb]{0.00,0.50,0.00}{##1}}}
\expandafter\def\csname PY@tok@kd\endcsname{\let\PY@bf=\textbf\def\PY@tc##1{\textcolor[rgb]{0.00,0.50,0.00}{##1}}}
\expandafter\def\csname PY@tok@kn\endcsname{\let\PY@bf=\textbf\def\PY@tc##1{\textcolor[rgb]{0.00,0.50,0.00}{##1}}}
\expandafter\def\csname PY@tok@kr\endcsname{\let\PY@bf=\textbf\def\PY@tc##1{\textcolor[rgb]{0.00,0.50,0.00}{##1}}}
\expandafter\def\csname PY@tok@bp\endcsname{\def\PY@tc##1{\textcolor[rgb]{0.00,0.50,0.00}{##1}}}
\expandafter\def\csname PY@tok@fm\endcsname{\def\PY@tc##1{\textcolor[rgb]{0.00,0.00,1.00}{##1}}}
\expandafter\def\csname PY@tok@vc\endcsname{\def\PY@tc##1{\textcolor[rgb]{0.10,0.09,0.49}{##1}}}
\expandafter\def\csname PY@tok@vg\endcsname{\def\PY@tc##1{\textcolor[rgb]{0.10,0.09,0.49}{##1}}}
\expandafter\def\csname PY@tok@vi\endcsname{\def\PY@tc##1{\textcolor[rgb]{0.10,0.09,0.49}{##1}}}
\expandafter\def\csname PY@tok@vm\endcsname{\def\PY@tc##1{\textcolor[rgb]{0.10,0.09,0.49}{##1}}}
\expandafter\def\csname PY@tok@sa\endcsname{\def\PY@tc##1{\textcolor[rgb]{0.73,0.13,0.13}{##1}}}
\expandafter\def\csname PY@tok@sb\endcsname{\def\PY@tc##1{\textcolor[rgb]{0.73,0.13,0.13}{##1}}}
\expandafter\def\csname PY@tok@sc\endcsname{\def\PY@tc##1{\textcolor[rgb]{0.73,0.13,0.13}{##1}}}
\expandafter\def\csname PY@tok@dl\endcsname{\def\PY@tc##1{\textcolor[rgb]{0.73,0.13,0.13}{##1}}}
\expandafter\def\csname PY@tok@s2\endcsname{\def\PY@tc##1{\textcolor[rgb]{0.73,0.13,0.13}{##1}}}
\expandafter\def\csname PY@tok@sh\endcsname{\def\PY@tc##1{\textcolor[rgb]{0.73,0.13,0.13}{##1}}}
\expandafter\def\csname PY@tok@s1\endcsname{\def\PY@tc##1{\textcolor[rgb]{0.73,0.13,0.13}{##1}}}
\expandafter\def\csname PY@tok@mb\endcsname{\def\PY@tc##1{\textcolor[rgb]{0.40,0.40,0.40}{##1}}}
\expandafter\def\csname PY@tok@mf\endcsname{\def\PY@tc##1{\textcolor[rgb]{0.40,0.40,0.40}{##1}}}
\expandafter\def\csname PY@tok@mh\endcsname{\def\PY@tc##1{\textcolor[rgb]{0.40,0.40,0.40}{##1}}}
\expandafter\def\csname PY@tok@mi\endcsname{\def\PY@tc##1{\textcolor[rgb]{0.40,0.40,0.40}{##1}}}
\expandafter\def\csname PY@tok@il\endcsname{\def\PY@tc##1{\textcolor[rgb]{0.40,0.40,0.40}{##1}}}
\expandafter\def\csname PY@tok@mo\endcsname{\def\PY@tc##1{\textcolor[rgb]{0.40,0.40,0.40}{##1}}}
\expandafter\def\csname PY@tok@ch\endcsname{\let\PY@it=\textit\def\PY@tc##1{\textcolor[rgb]{0.25,0.50,0.50}{##1}}}
\expandafter\def\csname PY@tok@cm\endcsname{\let\PY@it=\textit\def\PY@tc##1{\textcolor[rgb]{0.25,0.50,0.50}{##1}}}
\expandafter\def\csname PY@tok@cpf\endcsname{\let\PY@it=\textit\def\PY@tc##1{\textcolor[rgb]{0.25,0.50,0.50}{##1}}}
\expandafter\def\csname PY@tok@c1\endcsname{\let\PY@it=\textit\def\PY@tc##1{\textcolor[rgb]{0.25,0.50,0.50}{##1}}}
\expandafter\def\csname PY@tok@cs\endcsname{\let\PY@it=\textit\def\PY@tc##1{\textcolor[rgb]{0.25,0.50,0.50}{##1}}}

\def\PYZbs{\char`\\}
\def\PYZus{\char`\_}
\def\PYZob{\char`\{}
\def\PYZcb{\char`\}}
\def\PYZca{\char`\^}
\def\PYZam{\char`\&}
\def\PYZlt{\char`\<}
\def\PYZgt{\char`\>}
\def\PYZsh{\char`\#}
\def\PYZpc{\char`\%}
\def\PYZdl{\char`\$}
\def\PYZhy{\char`\-}
\def\PYZsq{\char`\'}
\def\PYZdq{\char`\"}
\def\PYZti{\char`\~}
% for compatibility with earlier versions
\def\PYZat{@}
\def\PYZlb{[}
\def\PYZrb{]}
\makeatother


    % For linebreaks inside Verbatim environment from package fancyvrb. 
    \makeatletter
        \newbox\Wrappedcontinuationbox 
        \newbox\Wrappedvisiblespacebox 
        \newcommand*\Wrappedvisiblespace {\textcolor{red}{\textvisiblespace}} 
        \newcommand*\Wrappedcontinuationsymbol {\textcolor{red}{\llap{\tiny$\m@th\hookrightarrow$}}} 
        \newcommand*\Wrappedcontinuationindent {3ex } 
        \newcommand*\Wrappedafterbreak {\kern\Wrappedcontinuationindent\copy\Wrappedcontinuationbox} 
        % Take advantage of the already applied Pygments mark-up to insert 
        % potential linebreaks for TeX processing. 
        %        {, <, #, %, $, ' and ": go to next line. 
        %        _, }, ^, &, >, - and ~: stay at end of broken line. 
        % Use of \textquotesingle for straight quote. 
        \newcommand*\Wrappedbreaksatspecials {% 
            \def\PYGZus{\discretionary{\char`\_}{\Wrappedafterbreak}{\char`\_}}% 
            \def\PYGZob{\discretionary{}{\Wrappedafterbreak\char`\{}{\char`\{}}% 
            \def\PYGZcb{\discretionary{\char`\}}{\Wrappedafterbreak}{\char`\}}}% 
            \def\PYGZca{\discretionary{\char`\^}{\Wrappedafterbreak}{\char`\^}}% 
            \def\PYGZam{\discretionary{\char`\&}{\Wrappedafterbreak}{\char`\&}}% 
            \def\PYGZlt{\discretionary{}{\Wrappedafterbreak\char`\<}{\char`\<}}% 
            \def\PYGZgt{\discretionary{\char`\>}{\Wrappedafterbreak}{\char`\>}}% 
            \def\PYGZsh{\discretionary{}{\Wrappedafterbreak\char`\#}{\char`\#}}% 
            \def\PYGZpc{\discretionary{}{\Wrappedafterbreak\char`\%}{\char`\%}}% 
            \def\PYGZdl{\discretionary{}{\Wrappedafterbreak\char`\$}{\char`\$}}% 
            \def\PYGZhy{\discretionary{\char`\-}{\Wrappedafterbreak}{\char`\-}}% 
            \def\PYGZsq{\discretionary{}{\Wrappedafterbreak\textquotesingle}{\textquotesingle}}% 
            \def\PYGZdq{\discretionary{}{\Wrappedafterbreak\char`\"}{\char`\"}}% 
            \def\PYGZti{\discretionary{\char`\~}{\Wrappedafterbreak}{\char`\~}}% 
        } 
        % Some characters . , ; ? ! / are not pygmentized. 
        % This macro makes them "active" and they will insert potential linebreaks 
        \newcommand*\Wrappedbreaksatpunct {% 
            \lccode`\~`\.\lowercase{\def~}{\discretionary{\hbox{\char`\.}}{\Wrappedafterbreak}{\hbox{\char`\.}}}% 
            \lccode`\~`\,\lowercase{\def~}{\discretionary{\hbox{\char`\,}}{\Wrappedafterbreak}{\hbox{\char`\,}}}% 
            \lccode`\~`\;\lowercase{\def~}{\discretionary{\hbox{\char`\;}}{\Wrappedafterbreak}{\hbox{\char`\;}}}% 
            \lccode`\~`\:\lowercase{\def~}{\discretionary{\hbox{\char`\:}}{\Wrappedafterbreak}{\hbox{\char`\:}}}% 
            \lccode`\~`\?\lowercase{\def~}{\discretionary{\hbox{\char`\?}}{\Wrappedafterbreak}{\hbox{\char`\?}}}% 
            \lccode`\~`\!\lowercase{\def~}{\discretionary{\hbox{\char`\!}}{\Wrappedafterbreak}{\hbox{\char`\!}}}% 
            \lccode`\~`\/\lowercase{\def~}{\discretionary{\hbox{\char`\/}}{\Wrappedafterbreak}{\hbox{\char`\/}}}% 
            \catcode`\.\active
            \catcode`\,\active 
            \catcode`\;\active
            \catcode`\:\active
            \catcode`\?\active
            \catcode`\!\active
            \catcode`\/\active 
            \lccode`\~`\~ 	
        }
    \makeatother

    \let\OriginalVerbatim=\Verbatim
    \makeatletter
    \renewcommand{\Verbatim}[1][1]{%
        %\parskip\z@skip
        \sbox\Wrappedcontinuationbox {\Wrappedcontinuationsymbol}%
        \sbox\Wrappedvisiblespacebox {\FV@SetupFont\Wrappedvisiblespace}%
        \def\FancyVerbFormatLine ##1{\hsize\linewidth
            \vtop{\raggedright\hyphenpenalty\z@\exhyphenpenalty\z@
                \doublehyphendemerits\z@\finalhyphendemerits\z@
                \strut ##1\strut}%
        }%
        % If the linebreak is at a space, the latter will be displayed as visible
        % space at end of first line, and a continuation symbol starts next line.
        % Stretch/shrink are however usually zero for typewriter font.
        \def\FV@Space {%
            \nobreak\hskip\z@ plus\fontdimen3\font minus\fontdimen4\font
            \discretionary{\copy\Wrappedvisiblespacebox}{\Wrappedafterbreak}
            {\kern\fontdimen2\font}%
        }%
        
        % Allow breaks at special characters using \PYG... macros.
        \Wrappedbreaksatspecials
        % Breaks at punctuation characters . , ; ? ! and / need catcode=\active 	
        \OriginalVerbatim[#1,codes*=\Wrappedbreaksatpunct]%
    }
    \makeatother

    % Exact colors from NB
    \definecolor{incolor}{HTML}{303F9F}
    \definecolor{outcolor}{HTML}{D84315}
    \definecolor{cellborder}{HTML}{CFCFCF}
    \definecolor{cellbackground}{HTML}{F7F7F7}
    
    % prompt
    \makeatletter
    \newcommand{\boxspacing}{\kern\kvtcb@left@rule\kern\kvtcb@boxsep}
    \makeatother
    \newcommand{\prompt}[4]{
        \ttfamily\llap{{\color{#2}[#3]:\hspace{3pt}#4}}\vspace{-\baselineskip}
    }
    

    
    % Prevent overflowing lines due to hard-to-break entities
    \sloppy 
    % Setup hyperref package
    \hypersetup{
      breaklinks=true,  % so long urls are correctly broken across lines
      colorlinks=true,
      urlcolor=urlcolor,
      linkcolor=linkcolor,
      citecolor=citecolor,
      }
    % Slightly bigger margins than the latex defaults
    
    \geometry{verbose,tmargin=1in,bmargin=1in,lmargin=1in,rmargin=1in}
    
    

\begin{document}
    
    \maketitle
    
    

    
    \hypertarget{programmering}{%
\section{Programmering}\label{programmering}}

    \hypertarget{hvorfor-skal-elevene-luxe6re-programmering}{%
\subsection{Hvorfor skal elevene lære
programmering?}\label{hvorfor-skal-elevene-luxe6re-programmering}}

\hypertarget{generelle-grunner}{%
\subsubsection{Generelle grunner}\label{generelle-grunner}}

\begin{itemize}
\tightlist
\item
  Grunnleggende forståelse for IKT (almenndannelse)
\item
  Viktig nasjonal interesse med økt rekruttering til IKT
\end{itemize}

    \hypertarget{pedagogiske-og-didaktiske-grunner}{%
\subsubsection{Pedagogiske og didaktiske
grunner}\label{pedagogiske-og-didaktiske-grunner}}

\begin{itemize}
\tightlist
\item
  Lettere å arbeide med realistiske problemstillinger
\item
  Utfordrer elevene til å tenke systematisk
\item
  Utvikler logisk forståelse og tankegang
\end{itemize}

    \hypertarget{vanlig-faglig-problemstilling-i-grunnskolen}{%
\subsection{Vanlig faglig problemstilling i
grunnskolen:}\label{vanlig-faglig-problemstilling-i-grunnskolen}}

\begin{itemize}
\tightlist
\item
  Vei fart tid
\item
  Lineære funksjoner
\end{itemize}

Litt kjedelig

    

    {[}figur 1{]}{[}./bilder/vei\_fart\_tid\_linear.png{]}

    \begin{itemize}
\item
  Ikke veldig spennende
\item
  Hva skjer når farta ikke er konstant?
\item
  Svar med klassisk matematikk krever gjerne integrasjon /
  differensiallikninger
\item
  Programmering kan gjøre problemet tilgjengelig for (flinke) elever
  tidlig.
\item
  Veldig plagsomt at klassisk matematikk ikke gjør problemet mer
  tilgjengelig
\end{itemize}

    Vi skal se på en løsning med konstant akselerasjon senere - fra dette
kan man også bygge modeller der akselerasjonen varierer!

    \hypertarget{andre-aktuelle-anvendelser}{%
\subsection{Andre aktuelle
anvendelser}\label{andre-aktuelle-anvendelser}}

\begin{itemize}
\tightlist
\item
  Programmere og kalibrere måleinstrumenter i naturfag og kunst og
  håndtverk (arduino, raspberry pi)
\item
  Programmere og kalibrere aktuatorer i naturfag og kunst og håndtverk
  (arduino, raspberry pi)
\item
  Jobbe med store datasett (eksempler på datasett:
  https://www.kaggle.com/datasets)
\item
  Muligheter for å kombinere naturfag, matematikk, kunst og håndtverk i
  større grad
\item
  Mer tverrfaglig arbeid!
\item
  Programmering som verktøy i alle fag
\end{itemize}

    \hypertarget{et-enkelt-program}{%
\section{Et enkelt program}\label{et-enkelt-program}}

Å se på dette programmet har to hovedmål * Vite hvordan du kan kjøre
python-programmer * Vite hvordan du skriver ut data (tall, tekst etc)
fra python ut til skjermen * Bli kjent med datatypen for tekst:
\emph{strenger}

    \begin{tcolorbox}[breakable, size=fbox, boxrule=1pt, pad at break*=1mm,colback=cellbackground, colframe=cellborder]
\prompt{In}{incolor}{1}{\boxspacing}
\begin{Verbatim}[commandchars=\\\{\}]
\PY{n+nb}{print}\PY{p}{(}\PY{l+s+s1}{\PYZsq{}}\PY{l+s+s1}{Hello, world!}\PY{l+s+s1}{\PYZsq{}}\PY{p}{)}
\end{Verbatim}
\end{tcolorbox}

    \begin{Verbatim}[commandchars=\\\{\}]
Hello, world!
    \end{Verbatim}

    \begin{itemize}
\tightlist
\item
  Funksjonen \texttt{print} skriver teksten
  \texttt{\textquotesingle{}Hello,\ world!\textquotesingle{}} ut til
  skjermen.
\item
  Hermetegnene forteller Python at innholdet er tekst (en streng).
\item
  I programmering kaller vi data som skal være tekst for
  \textbf{strenger}.
\end{itemize}

    \hypertarget{et-program-som-behandler-tall}{%
\section{Et program som behandler
tall}\label{et-program-som-behandler-tall}}

    \begin{tcolorbox}[breakable, size=fbox, boxrule=1pt, pad at break*=1mm,colback=cellbackground, colframe=cellborder]
\prompt{In}{incolor}{4}{\boxspacing}
\begin{Verbatim}[commandchars=\\\{\}]
\PY{c+c1}{\PYZsh{} Regn ut arealet av et pararellogram med lengde l og bredde h}
\PY{n}{l} \PY{o}{=} \PY{l+m+mi}{8}
\PY{n}{h} \PY{o}{=} \PY{l+m+mf}{3.5}

\PY{n}{A} \PY{o}{=} \PY{n}{l}\PY{o}{*}\PY{n}{h}
\PY{n+nb}{print}\PY{p}{(}\PY{n}{A}\PY{p}{)}
\end{Verbatim}
\end{tcolorbox}

    \begin{Verbatim}[commandchars=\\\{\}]
28.0
    \end{Verbatim}

    Programmet regner ut arealet av et pararellogram - vi kan tenke på
programmet som en oppskrift for dette.

\begin{itemize}
\item
  Den øverste linjen er en kommentar. Den blir ignorert av datamaskinen
  og skal gjøre programmet lettere å lese for mennesker
\item
  \texttt{l} og \texttt{h} er \emph{variabler} av typen \texttt{int}
  (heltall) og \texttt{float} (et ``desimaltall''). En variabel kan du
  tenke på som en boks som inneholder data, og navnet på variabelen gir
  oss en måte å få tak i dataene
\item
  Til slutt regner vi ut produktet \texttt{l*h}, lagrer det i variabelen
  A og skriver det ut til skjermen
\end{itemize}

    Gode variabelnavn og gode kommentarer er viktig for lesbar kode. Les mer
om gode variabelnavn og kommentarer her: http://lstor.me/kommentarer/.

    

    \hypertarget{forbedring-av-programmet}{%
\section{Forbedring av programmet}\label{forbedring-av-programmet}}

Programmet har en svakhet: hver gang du vil endre dimensjonene på
pararellogrammet, må koden endres.

Bedre løsning: les inn tallene fra brukeren

    \begin{tcolorbox}[breakable, size=fbox, boxrule=1pt, pad at break*=1mm,colback=cellbackground, colframe=cellborder]
\prompt{In}{incolor}{2}{\boxspacing}
\begin{Verbatim}[commandchars=\\\{\}]
\PY{c+c1}{\PYZsh{} Regn ut arealet av et pararellogram med lengde l og bredde h}
\PY{n}{l} \PY{o}{=} \PY{n+nb}{input}\PY{p}{(}\PY{l+s+s1}{\PYZsq{}}\PY{l+s+s1}{Skriv inn lengden på pararellogrammet: }\PY{l+s+s1}{\PYZsq{}}\PY{p}{)}
\PY{n}{h} \PY{o}{=} \PY{n+nb}{input}\PY{p}{(}\PY{l+s+s1}{\PYZsq{}}\PY{l+s+s1}{Skriv inn høyden til pararellogrammet: }\PY{l+s+s1}{\PYZsq{}}\PY{p}{)}

\PY{c+c1}{\PYZsh{} Gjør strengene `l` og `h` om til desimaltall}
\PY{n}{l} \PY{o}{=} \PY{n+nb}{float}\PY{p}{(}\PY{n}{l}\PY{p}{)}
\PY{n}{h} \PY{o}{=} \PY{n+nb}{float}\PY{p}{(}\PY{n}{h}\PY{p}{)}

\PY{n}{A} \PY{o}{=} \PY{n}{l}\PY{o}{*}\PY{n}{h}
\PY{n+nb}{print}\PY{p}{(}\PY{n}{A}\PY{p}{)}
\end{Verbatim}
\end{tcolorbox}

    \begin{Verbatim}[commandchars=\\\{\}]
Skriv inn lengden på pararellogrammet: 4.0
Skriv inn høyden til pararellogrammet: 5,0
    \end{Verbatim}

    \begin{Verbatim}[commandchars=\\\{\}]

        ---------------------------------------------------------------------------

        ValueError                                Traceback (most recent call last)

        <ipython-input-2-56b59ba95142> in <module>
          5 \# Gjør strengene `l` og `h` om til desimaltall
          6 l = float(l)
    ----> 7 h = float(h)
          8 
          9 A = l*h


        ValueError: could not convert string to float: '5,0'

    \end{Verbatim}

    \hypertarget{oppgaver-for-uxe5-komme-i-gang-med-python}{%
\section{Oppgaver for å komme i gang med
python}\label{oppgaver-for-uxe5-komme-i-gang-med-python}}

    \hypertarget{oppgave-1}{%
\subsection{Oppgave 1}\label{oppgave-1}}

Skriv et program \texttt{hei\_verden.py} som skriver teksten
\texttt{Hei,\ Verden!} ut til skjermen.

    \hypertarget{oppgave-2}{%
\subsection{Oppgave 2}\label{oppgave-2}}

Skriv et program som regner ut arealet til en trekant med lengde \(h\)
og grunnlinje \(g\). Du kan velge verdier for \(h\) og \(g\) selv.

    \hypertarget{oppgave-3}{%
\subsection{Oppgave 3}\label{oppgave-3}}

Gjør om programmet over slik at det leser inn lengden \(h\) og
grunnlinja \(g\), regner ut arealet til trekanten og skriver det ut til
skjermen.

    \hypertarget{oppgave-4-lag-et-didaktisk-verktuxf8y}{%
\subsection{Oppgave 4: Lag et didaktisk
verktøy}\label{oppgave-4-lag-et-didaktisk-verktuxf8y}}

Du skal introdusere elevene dine for funksjoner for første gang, og
ønsker å leke den (for lærere) kjente leken ``gjett regelen'' der
elevene sier noen tall, du beregner en tilhørende verdi med en lineær
funksjon, og elevene skal prøve å formulere hva funksjonen gjør med
tallene.

Bruk en lineær funksjon \(y = ax + b\) der du selv bestemmer tallene
\(a\) og \(b\) som holdes hemmelig for elevene. Programmet skal lese inn
en verdi for \(x\) og gi ut verdien for \(y\).

    \hypertarget{oppgave-5}{%
\subsection{Oppgave 5}\label{oppgave-5}}

En båt seiler med konstant hastighet \(v= 43.2 \text{km/t}\). Fra havn
\(A\) til havn \(B\). Ved tiden \(t=0\) var båten \(180 \text{ km}\)
unna havn \(A\).

Skriv et program som regner ut strekningen \(s_A(t)\) fra havn \(A\)
etter tiden \(t\).

    \hypertarget{oppgave-6}{%
\subsection{Oppgave 6}\label{oppgave-6}}

Modifiser programmet ditt i oppgave 3 slik at det leser inn farten
\(v\), strekningen \(s_0\) ved tiden \(t=0\) og tiden \(t_0\).
Programmet skal regne ut strekningen \(s\) etter tiden \(t\) og skrive
det ut til skjermen.

    \hypertarget{oppgave-7-bruk-en-komplisert-formel}{%
\subsection{Oppgave 7: Bruk en komplisert
formel}\label{oppgave-7-bruk-en-komplisert-formel}}

Programmer kan gjøre det lettere å beregne verdien til kompliserte
utrykk. Et objekt som beveger seg gjennom en væske eller en gass med
tetthet \(\rho\) målt i \(\text{kg/m}^3\) og dynamisk viskositet \(\mu\)
målt i \(\text{kg/ms}\), opplever en drag-kraft

\[ F_d = \frac{1}{2}C_dA\rho v^2, \]

der \(C_d\) er \emph{drag-koeffisienten} som avhenger av objektets form,
overflate osv., og \(A\) er arealet av objektets tverrsnitt målt i
kvadratmeter. Du skal beregne \(F_d\) når du har fått oppgitt følgende
data:

\begin{align*}
\mu & = 1.802\cdot10^{-5} & \\
\rho & = 1.225 & \\
C_d & = 0.8 & \\
A & = 0.8 & \\
\end{align*}

\textbf{a)} Lag et program \texttt{drag\_koeffisient.py} som beregner
drag-koeffisienten \(F_d\).

\textbf{b)} Gjør samme beregningen i et ipython-shell.

    \hypertarget{noen-konkrete-anvendelser}{%
\section{Noen konkrete anvendelser}\label{noen-konkrete-anvendelser}}

    Vi kan begynne med å se på hvordan man kan gjøre beregninger knyttet til
bevegelse under konstant akselerasjon \(a\). Det finnes en formel
programmet kan testes mot: \[ s(t) = s_0 + v_0t + \frac{1}{2}at^2 \]

    

    \hypertarget{eksempel-puxe5-algoritmisk-tenkning}{%
\subsubsection{Eksempel på algoritmisk
tenkning}\label{eksempel-puxe5-algoritmisk-tenkning}}

Får å beregne strekningen \(s\), må vi bryte problemet opp i mange
delproblemer der vi beregner hastigheten \(v\) fra akselerasjonen \(a\)
i mange små tidssteg, og i hvert tidssteg oppdaterer vi \(s\) med
formlene

\begin{align*}
v(t_n) = 10 - at_n \\[1.2em]
s_{n+1} = s_{n} + \Delta t \cdot v(t_n)
\end{align*}

    \begin{tcolorbox}[breakable, size=fbox, boxrule=1pt, pad at break*=1mm,colback=cellbackground, colframe=cellborder]
\prompt{In}{incolor}{7}{\boxspacing}
\begin{Verbatim}[commandchars=\\\{\}]
\PY{k}{def} \PY{n+nf}{modell\PYZus{}strekning}\PY{p}{(}\PY{n}{T}\PY{p}{,} \PY{n}{s\PYZus{}0}\PY{p}{,} \PY{n}{v}\PY{p}{)}\PY{p}{:}
    \PY{l+s+sd}{\PYZdq{}\PYZdq{}\PYZdq{}}
\PY{l+s+sd}{    modell\PYZhy{}problem med kjent fart}

\PY{l+s+sd}{    idé: Hvis man kjenner s(t) og v(t), kan s(t + dt) tilnærmes slik:}
\PY{l+s+sd}{    s(t + dt) = s(t) + dt*v(t)}
\PY{l+s+sd}{    Jo mindre tidssteg dt, jo mer nøyaktig blir tilnærmingen.}
\PY{l+s+sd}{    \PYZdq{}\PYZdq{}\PYZdq{}}

    \PY{n}{dt} \PY{o}{=} \PY{l+m+mf}{0.1}   \PY{c+c1}{\PYZsh{} tidssteg}
    \PY{n}{t} \PY{o}{=} \PY{p}{[}\PY{l+m+mi}{0}\PY{p}{]}    \PY{c+c1}{\PYZsh{} første tidspunkt}
    \PY{n}{s} \PY{o}{=} \PY{p}{[}\PY{n}{s\PYZus{}0}\PY{p}{]}  \PY{c+c1}{\PYZsh{} start\PYZhy{}strekning}
    \PY{n}{k} \PY{o}{=} \PY{l+m+mi}{0}      \PY{c+c1}{\PYZsh{} telle\PYZhy{}variabel}

    \PY{k}{while} \PY{n}{t}\PY{p}{[}\PY{n}{k}\PY{p}{]} \PY{o}{\PYZlt{}} \PY{n}{T}\PY{p}{:}
        \PY{n}{s\PYZus{}next} \PY{o}{=} \PY{n}{s}\PY{p}{[}\PY{n}{k}\PY{p}{]} \PY{o}{+} \PY{n}{dt}\PY{o}{*}\PY{n}{v}\PY{p}{(} \PY{n}{t}\PY{p}{[}\PY{n}{k}\PY{p}{]} \PY{p}{)}
        \PY{n}{t\PYZus{}next} \PY{o}{=} \PY{n}{t}\PY{p}{[}\PY{n}{k}\PY{p}{]} \PY{o}{+} \PY{n}{dt}

        \PY{n}{s}\PY{o}{.}\PY{n}{append}\PY{p}{(}\PY{n}{s\PYZus{}next}\PY{p}{)}
        \PY{n}{t}\PY{o}{.}\PY{n}{append}\PY{p}{(}\PY{n}{t\PYZus{}next}\PY{p}{)}
        \PY{n}{k} \PY{o}{=} \PY{n}{k} \PY{o}{+} \PY{l+m+mi}{1}

    \PY{k}{return} \PY{n}{t}\PY{p}{,} \PY{n}{s}
\end{Verbatim}
\end{tcolorbox}

    \hypertarget{tidsperspektiv-puxe5-slike-prosjekter}{%
\subsubsection{Tidsperspektiv på slike
prosjekter}\label{tidsperspektiv-puxe5-slike-prosjekter}}

Min antakelse: én til to uker hvis elevene er kjent med bruk av løkker.

    \hypertarget{luxe6ringsverdi}{%
\subsubsection{Læringsverdi}\label{luxe6ringsverdi}}

\begin{itemize}
\tightlist
\item
  Algoritmisk tenkning
\item
  Vurdering av modellen, hvor stor må \(\Delta t\) være, etc
\item
  Går det an å utvide modellen? F.eks i tilknytning til Newtons lover i
  fysikk.
\item
  Hva skjer dersom vi utvider modellen og bruker den på en fjær som
  svinger?
\end{itemize}

    \hypertarget{behandling-av-store-datasett}{%
\subsubsection{Behandling av store
datasett}\label{behandling-av-store-datasett}}

Vi skal lese inn \textbf{digre} datasett. Regneark sliter med å behandle
så store mengder data, og det er også ganske knotete å holde på med. Vi
bruker pakken pandas. Dere kan bruke conda til å installere den. Skriv
dette i terminalen for å installere nødvendige pakker

\begin{verbatim}
conda install pandas
conda install xlrd
\end{verbatim}

    I mappen
\texttt{programmering\_uskole/programmer/eksempler\_datalogging} ligger
det en exel-fil \texttt{temperatur\_data.xlsx} med dummy-temperaturer.

    Vi kan bruke pandas til å både skrive data til excel-filer, og for å
laste data fra excel-filer inn i python. Du kan senere ta en titt på
hvordan dette er gjort i filene
\texttt{programmering\_uskole/programmer/eksempler\_datalogging/makedata.py}
og
\texttt{programmering\_uskole/programmer/eksempler\_datalogging/read\_data.py}

    \hypertarget{andre-anvendelser}{%
\subsection{Andre anvendelser}\label{andre-anvendelser}}

Biologi: Simulere bestander eller økosystemer. Start med eksponentiell
vekst, introduser bære-evne, introduser sesonger, introduser rovdyr,
introduser tilfeldige sykdommmer. Store muligheter!

Fysikk: Vei-fart-tid og newtons lover.

Tverrfaglige prosjekter: Naturfag, kunst og håndtverk, matematikk.
Utforske bruk av arduino og raspberry pi med måleinstrumenter og
aktuatorer. Sammenføyning som plastsveising og lodding aktuelt.

Bruk av pakken pandas kan være aktuelt i andre fag? (databehandling ut
over excel)

Personlig mening: synes det er lettere å bruke pandas enn excel \ldots{}

Viktig å lære en del grunnleggende emner først. Men aller viktigst å ha
det gøy!

    \hypertarget{viktigst-ha-det-guxf8y.-la-elevene-ha-det-guxf8y-med-koding}{%
\section{Viktigst: Ha det gøy. La elevene ha det gøy med
koding}\label{viktigst-ha-det-guxf8y.-la-elevene-ha-det-guxf8y-med-koding}}

Bruk programmering som et didaktisk verktøy. La elevene utvikle
ferdighetene gradvis. Tør å la elevene formulere egne prosjekter, og
tilby både utfordrende og enklere prosjekter elevene kan arbeide med.


    % Add a bibliography block to the postdoc
    
    
    
\end{document}
